\documentclass[reprint,english,notitlepage]{revtex4-1}  % defines the basic parameters of the document

% if you want a single-column, remove reprint

% allows special characters (including æøå)
\usepackage[utf8]{inputenc}
\usepackage[english]{babel}

%% note that you may need to download some of these packages manually, it depends on your setup.
%% I recommend downloading TeXMaker, because it includes a large library of the most common packages.

\usepackage{physics,amssymb}  % mathematical symbols (physics imports amsmath)
\usepackage{graphicx}         % include graphics such as plots
\graphicspath{{figs/}} %Setting the graphicspath
\numberwithin{equation}{section}
\def\thesection{\arabic{section}}

\usepackage{xcolor}           % set colors
\usepackage{hyperref}         % automagic cross-referencing (this is GODLIKE)
\usepackage{tikz}             % draw figures manually
\usepackage{listings}         % display code
\usepackage{subfigure}        % imports a lot of cool and useful figure commands
\usepackage{pythonhighlight}

% defines the color of hyperref objects
% Blending two colors:  blue!80!black  =  80% blue and 20% black
\hypersetup{ % this is just my personal choice, feel free to change things
    colorlinks,
    linkcolor={red!50!black},
    citecolor={blue!50!black},
    urlcolor={blue!80!black}}

%% Defines the style of the programming listing
%% This is actually my personal template, go ahead and change stuff if you want
\lstset{ %
	inputpath=,
	backgroundcolor=\color{white!88!black},
	basicstyle={\ttfamily\scriptsize},
	commentstyle=\color{magenta},
	language=Python,
	morekeywords={True,False},
	tabsize=4,
	stringstyle=\color{green!55!black},
	frame=single,
	keywordstyle=\color{blue},
	showstringspaces=false,
	columns=fullflexible,
	keepspaces=true}


%% USEFUL LINKS:
%%
%%   UiO LaTeX guides:        https://www.mn.uio.no/ifi/tjenester/it/hjelp/latex/
%%   mathematics:             https://en.wikibooks.org/wiki/LaTeX/Mathematics

%%   PHYSICS !                https://mirror.hmc.edu/ctan/macros/latex/contrib/physics/physics.pdf

%%   the basics of Tikz:       https://en.wikibooks.org/wiki/LaTeX/PGF/TikZ
%%   all the colors!:          https://en.wikibooks.org/wiki/LaTeX/Colors
%%   how to draw tables:       https://en.wikibooks.org/wiki/LaTeX/Tables
%%   code listing styles:      https://en.wikibooks.org/wiki/LaTeX/Source_Code_Listings
%%   \includegraphics          https://en.wikibooks.org/wiki/LaTeX/Importing_Graphics
%%   learn more about figures  https://en.wikibooks.org/wiki/LaTeX/Floats,_Figures_and_Captions
%%   automagic bibliography:   https://en.wikibooks.org/wiki/LaTeX/Bibliography_Management  (this one is kinda difficult the first time)
%%   REVTeX Guide:             http://www.physics.csbsju.edu/370/papers/Journal_Style_Manuals/auguide4-1.pdf
%%
%%   (this document is of class "revtex4-1", the REVTeX Guide explains how the class works)


%% CREATING THE .pdf FILE USING LINUX IN THE TERMINAL
%%
%% [terminal]$ pdflatex template.tex
%%
%% Run the command twice, always.
%% If you want to use \footnote, you need to run these commands (IN THIS SPECIFIC ORDER)
%%
%% [terminal]$ pdflatex template.tex
%% [terminal]$ bibtex template
%% [terminal]$ pdflatex template.tex
%% [terminal]$ pdflatex template.tex
%%
%% Don't ask me why, I don't know.
\newcommand{\TT}{\textsf{T}}
\newcommand{\LL}{\textsf{L}}
\newcommand{\MM}{\textsf{M}}

\begin{document}
\title{AST3220 - Project 3: \\ Inflation without approximation}   % self-explanatory
\author{Candidate nr. 14}
\date{\today}                             % self-explanatory
\noaffiliation                            % ignore this
               % marks the end of the abstract
\maketitle                                % creates the title, author, date & abstract

% the fundamental components of scientific reports:
\section{Problem a)}
We will in this section let $\LL$, $\TT$ and $\MM$ represent dimensions length, time
and mass respectively.
\\
\\
As $H_i$ necessarily has the same dimensions as $H$ it is immediately
clear that $h$ must be dimensionless:
\begin{align}
	[h] = [H] [H_i]^{-1} = 1
\end{align}
To check the units of $\tau$ we first examine the dimensions of the Hubble
parameter. The Hubble parameter has dimension velocity per distance, which can
be written
\begin{align}
	[H] = (\LL \TT^{-1}) \LL^{-1} = \TT^{-1}
\end{align}
so that $\tau$ is dimensionless as well:
\begin{align}
  [\tau] = [H] [t] = \TT^{-1} \TT = 1
\end{align}
Both $\phi$ and the Planck energy $E_p$ has units of energy, making $\psi$ dimensionless
as well:
\begin{align}
	[\psi] = [\phi] [E_p]^{-1} = \MM \LL^2 \TT^{-2} (\MM \LL^2 \TT^{-2})^{-1} = 1
\end{align}
Lastly, we rewrite the potential $v$ using the definition $E_p^2 = \hbar c^5/G$,
so that
\begin{align}
	v = \frac{\hbar c^3}{H_i^2 E_p^2} = \frac{1}{H_i^2} \frac{G}{c^2}V
\end{align}
Using the definition of $H_i$, we then see that
\begin{align}
	[v] = [H_i]^{-1} [G c^{-2} V] = [H_i]^{-1}[H_i] = 1
\end{align}
so $v$ is dimensionless as well

\section{Problem b)}
\subsubsection{Hubble parameter and scale factor}
Using the definitions of the dimensionless variables and applying the chain rule,
we see that
\begin{align}
	\frac{d}{d\tau}\left(\ln \frac{a}{a_i}\right)
				&= \frac{dt}{d\tau}\frac{d}{dt}\left(\ln \frac{a}{a_i}\right) \\
				&= \frac{1}{H_i} \frac{\dot{a}}{a} \\
				&= \frac{H}{H_i} \\
				&= h
\end{align}
\subsubsection{Equation of motion}
Similarly, we continue using the chain rule to rewrite $\dot{\phi}$,
$\ddot{\phi}$ and $V$' in terms of $\tau$ and $\psi$:
\begin{align}
	\frac{d\phi}{dt} &= \frac{d\tau}{dt}\frac{d\phi}{d\psi}\frac{d\psi}{d\tau} \label{eq:dphidt}
									 = H_i E_p \frac{d\psi}{d\tau} \\
  \frac{d^2\phi}{dt^2} &= \frac{d\tau}{dt}\frac{d}{d\tau}\frac{d\phi}{dt}
									 = H_i^2 E_p \frac{d^2\psi}{d\tau^2} \\
	\frac{dV}{d\phi} &= \frac{d\psi}{d\phi}\frac{dV}{dv}\frac{dv}{d\psi}
									 = \frac{1}{E_p} \frac{H_i^2 E_p^2}{\hbar c^3} \frac{dv}{d\psi}
\end{align}
Which can be inserted into the equation of motion, giving
\begin{align}
	H_i^2 E_p \frac{d^2\psi}{d\tau^2} + 3H H_i E_p \frac{d\psi}{d\tau}
				+ H_i^2 E_p \frac{dv}{d\psi} = 0
\end{align}
Dividing both sides by $H_i^2 E_p$ and using the definition $h=H/H_i$, this
reduces to
\begin{align}
	\frac{d^2\psi}{d\tau^2} + 3h\frac{d\psi}{d\tau} + \frac{dv}{d\psi} = 0 \label{eq:eom}
\end{align}
\subsubsection{First Friedmann equation}
Lastly, we rewrite the first Friedmann equation in terms of the dimensionless quantities.
We have
\begin{align}
	H^2 &= \frac{8\pi G}{3 c^2}\left[ \frac{1}{2\hbar c^3}\dot{\phi} + V(\phi)\right] \\
			&= \frac{8\pi G}{3 c^2}\left[ \frac{H_i^2 E_p^2}{2\hbar c^3}\left(\frac{d\psi}{d\tau}\right)^2 + V(\phi)\right] \\
			&= \frac{8\pi G}{3 c^2}\frac{H_i^2 E_p^2}{\hbar c^3}\left[\frac{1}{2}\left(\frac{d\psi}{d\tau}\right)^2 + v(\psi)\right]
\end{align}
Where we used equation (\ref{eq:dphidt}) and the definition of $v$. Next,
inserting the definition of the Planck energy
\begin{align}
	E_p^2 = \frac{\hbar c^5}{G}
\end{align}
and dividing both sides by $H_i$, we finally get:
\begin{align}
	h^2 = \frac{8\pi}{3}\left[ \frac{1}{2}\left(\frac{d\psi}{d\tau}\right)^2 + v(\psi)\right] \label{eq:h}
\end{align}
\section{Problem c)}
In the slow-roll regime, $\frac{d^2\psi}{d\tau^2}$ is small
\begin{align}
	\frac{d^2\psi}{d\tau^2} \ll \frac{dv}{d\psi}
\end{align}
Inserting, the equation of motion can then be approximated as
\begin{align}
	3h\frac{d\psi}{d\tau} + \frac{dv}{d\psi} = 0
\end{align}
This is a first-order differential equation, which only requires an initial
value for $\psi$. Thus, when in the slow-roll approximation, the initial value
of $\frac{d\psi}{d\tau}$ does not matter as long as it fulfils the
slow-roll conditions.

\section{Problem d)}
During slow-roll, the number of remaining e-folds until inflation ends can be
calculated as
\begin{align}
	N &= \frac{8\pi}{E_p^2} \int_{\phi_{\mathrm{end}}}^\phi \frac{V}{V'}\ d\phi
\end{align}
Substituting $d\phi = E_p d\psi$, we then have
\begin{align}
	\frac{V}{V'} = \frac{1}{2}\phi = \frac{E_p}{2}\psi
\end{align}
Giving
\begin{align}
	N &= 4\pi \int_{\psi_{\mathrm{end}}}^\psi \psi\ d\psi \\
	  &= 2\pi \bigg( \psi^2 - \psi_{\mathrm{end}}^2 \bigg)	\label{eq:N-psi_i}
\end{align}
where $\psi_{\mathrm{end}}$ is given by taking $\epsilon(\phi_{\mathrm{end}})=1$
as the end of inflation:
\begin{align}
	\epsilon(\phi_{\mathrm{end}}) = \frac{E_p^2}{16\pi} \left(\frac{V'}{V}\right)^2
	 	&= \frac{1}{16\pi}\left(\frac{\frac{dv}{d\psi}}{v}\right) \\
											 &= \frac{1}{4\pi \psi_{\mathrm{end}}^2} \\
											 &= 1 \\
\end{align}
Solving for $\psi_end$ this gives
\begin{align}
	\Rightarrow \ \ \ \ \psi_{\mathrm{end}}^2 &= \frac{1}{4\pi}
\end{align}
Inserting this, the number of remaining e-folds is
\begin{align}
	N = 2\pi \psi^2 - \frac{1}{2}
\end{align}
The number of remaining e-folds at the initial time $t=t_i$ is
the total number of e-folds $N_{\mathrm{tot}}$. Evaluating $N(t)$ at $t_i$ and solving for
the initial field value $\psi_i$ we then get:
\begin{align}
	\psi_i = \sqrt{ \frac{1}{2\pi} \left(N_{\mathrm{tot}} + \frac{1}{2} \right) }
\end{align}
Which evaluates to $\psi_i \approx 8.925$ for $500$ total e-folds of inflation.

\section{Problem e) \& f)}
To solve the equations of motion of the field, we rename
$\xi = \frac{d\psi}{d\tau}$ so equation \ref{eq:eom} can instead be written as
a set of two first-order equations:
\begin{align}
	\frac{d\psi}{d\tau} &= \xi \label{eq:eom_1} \\
	\frac{d\xi}{d\tau}  &= -3h(\xi, \psi) \xi - \frac{d}{d\psi}v(\psi) \label{eq:eom_2}
\end{align}
where $h$ is again given by the first Friedmann equation
\begin{align}
	h^2 = \frac{8\pi}{3}\left[ \frac{1}{2}\xi^2 + v(\psi)\right]
\end{align}
and $v(\psi)$ is given by our choice of potential. Equations \ref{eq:eom_1} and
\ref{eq:eom_2} are then easily solved using a numerical integrator. A plot of
the resulting field $\psi$ plotted against $\tau$ is attached in Figure
(\ref{fig:quad_psi}), along with the slow-roll solution.
\\ \\
With the
solutions $\xi$ and $\psi$ we can easily find $\ln\left(\frac{a}{a_i}\right)$
by numerically evaluating the integral
\begin{align}
	\ln\left(\frac{a}{a_i}\right) &= \int_0^t H dt \\
																&= \int_0^\tau h(\xi, \psi) d\tau
\end{align}
The results are plotted in Figure (\ref{fig:quad_a}), along with the slow-roll
solution
\\ \\
Based on the lectures, we expect the field to decay until it reaches $\psi_{\mathrm{end}}$.
When it reaches $\psi_{\mathrm{end}}$ it should then start behaving like a damped oscillator
centred around $\psi=0$. This is in agreement with the numerical solution
(Figure \ref{fig:quad_psi}), which shows the field decaying linearly until it
enters the oscillating phase at $\tau\approx 1000$, indicating the end of inflation.
This is also when the slow-roll approximation starts deviating significantly
from the exact solution, as it instead continues decaying at a constant rate.
\\ \\
This is around the same time the slow-roll approximation for the scale factor
deviates from the exact solution (Figure \ref{fig:quad_a}), as one expects.
While the exact numerical solution shows the expansion halting as inflation ends,
the slow-roll approximation continues expanding shortly after.

\begin{figure}[h!]
	\includegraphics[width=\linewidth]{QuadraticPotential_field-value.png}
	\caption{The dimensionless scalar field $\psi$ for the quadratic potential,
	along with its approximation in the slow-roll regime. Plotted against the
	dimensionless time $\tau$.}
	\label{fig:quad_psi}
\end{figure}

\begin{figure}[h!]
	\includegraphics[width=\linewidth]{QuadraticPotential_scale-factor.png}
	\caption{The logarithm of the scale factor $\ln\left(\frac{a}{a_i}\right)$
	for a quadratic potential, along with the slow-roll approximation.
	Plotted as a function of the dimensionless time $\tau$.}
	\label{fig:quad_a}
\end{figure}

\section{Problem g)}
The slow-roll parameters are plotted against $\tau$ in Figure
(\ref{fig:QuadraticPotential_slowroll-tau}), along with the slow-roll
approximations.
For the quadratic potential, the slow-roll parameters are
\begin{align}
	\epsilon = \eta = \frac{1}{4 \pi}\frac{1}{\psi^2}
\end{align}
To find the total number of e-folds, we  numerically integrated
\begin{align}
	N_{\mathrm{tot}} = \int_{t_{\mathrm{end}}}^{t_i} H\ dt = \int_{\tau_{\mathrm{end}}}^{\tau_i} h\ d\tau
\end{align}
where the boundary $\tau_{\mathrm{end}}$ is taken care of by cutting off the solution
arrays $\tau$ and $h$ where $\epsilon=1$. This gives a total
\begin{align}
	N_{\mathrm{tot}} = 501.4
\end{align}
$e$-foldings produced during inflation, slightly more than the $500$ e-foldings
predicted by the slow-roll approximation.
\begin{figure}[h!]
	\includegraphics[width=\linewidth]{QuadraticPotential_slowroll-tau.png}
	\caption{The slow-roll parameters $\epsilon$ and $\eta$ for a quadratic potential,
	plotted against dimensionless time $\tau$.}
	\label{fig:QuadraticPotential_slowroll-tau}
\end{figure}

\section{Problem h)}
We now want to express the equation of state parameter in terms of the
dimensionless variables. The equation of state of the scalar field is given by
\begin{align}
 w_\phi =	\frac{p_\phi}{\rho_\phi c^2}
	= \frac{ \frac{1}{2\hbar c^3}\left(\frac{d\phi}{dt}\right)^2 - V }
				 { \frac{1}{2\hbar c^3}\left(\frac{d\phi}{dt}\right)^2 + V }
\end{align}
Earlier, we found that
\begin{align}
	\frac{d\phi}{dt} = H_i E_p \frac{d\psi}{d\tau}
\end{align}
and by definition we have
\begin{align}
	V = \frac{H_i^2 E_p^2}{\hbar c^3} v
\end{align}
Inserting these into the equation of state, we then get
\begin{align}
	w_\phi
	 = \frac{\ \frac{H_i^2E_p^2}{2\hbar c^3} \ }{\frac{H_i^2E_p^2}{2\hbar c^3}}
	 \frac{ \frac{1}{2}\left(\frac{d\psi}{d\tau}\right)^2 - v }
					{ \frac{1}{2}\left(\frac{d\psi}{d\tau}\right)^2	+ v }
	 = \frac{ \frac{1}{2}\left(\frac{d\psi}{d\tau}\right)^2 - v }
 				 { \frac{1}{2}\left(\frac{d\psi}{d\tau}\right)^2 + v }
\end{align}
which is what we wanted to show.
\section{Problem i)}
In the slow-roll regime we expect a small $\frac{d\psi}{d\tau}\ll1$, giving
$\left(\frac{d\psi}{d\tau}\right)^2 \approx 0$. The equation of state then
becomes
\begin{align}
	w_\phi \approx \frac{-v}{v} = -1
\end{align}
However, this should change when the field enters the oscillating phase.
Initially, when the field reaches $psi=0$, the potential becomes $v(0) =0$ which
gives an equation of state
\begin{align}
	w_\phi = \frac{\ \frac{1}{2}\left(\frac{d\psi}{d\tau}\right)^2 \ }{\frac{1}{2}\left(\frac{d\psi}{d\tau}\right)^2} = 1
\end{align}
However, the field will soon after reach a small peak where $\frac{d\psi}{d\tau}$
has to be zero, again giving an equation of state
\begin{align}
	w_\phi = \frac{-v}{v} = -1
\end{align}
As the field oscillates, this behaviour should repeat and we get an equation of
state oscillating between $w_\phi=-1$ and $w_\phi=1$. This is exactly what we
see in the numerical results seen in Figure (\ref{fig:QuadraticPotential_field-eos}).
\begin{figure}[h!]
	\includegraphics[width=\linewidth]{QuadraticPotential_field-eos.png}
	\caption{The scalar field $\phi$'s equation of state, given a quadratic
	potential, plotted against the dimensionless time $\tau$.}
	\label{fig:QuadraticPotential_field-eos}
\end{figure}

\section{Problem j)}
Plotting the slow-roll parameters against $N$ we get the plot shown in Figure
(\ref{fig:quadratic_slowroll-N}). As expected we see that the exact numerical
solution of $\epsilon$ ($=\eta$) approaches $\epsilon=1$ as $N$ approaches zero
and inflation ends.
\begin{figure}[h!]
	\includegraphics[width=\linewidth]{QuadraticPotential_slowroll-N.png}
	\caption{}
	\label{fig:quadratic_slowroll-N}
\end{figure}

\section{Problem k)}
The predicted curve in the $n$-$r$ plane for
$N\in[50,60]$ can be seen in Figure (\ref{fig:quadratic_slowroll-nr}).
The prediction shows a spectral index $n$ in the approximate range of $0.960$ to
$0.966$, and a tensor-to-scalar ratio $r$ in the approximate range
$0.135$ to $0.160$.
\begin{figure}[h!]
	\includegraphics[width=\linewidth]{QuadraticPotential_slowroll-nr.png}
	\caption{The spectral index $n$ and the tensor-to-scalar ratio of the
	quadratic potential, for $N\in[50,60]$.}
	\label{fig:quadratic_slowroll-nr}
\end{figure}

\section{Problem l)} \label{section:problem_l}
Substituting $y=-\sqrt{\frac{2}{3}}\frac{\phi}{M_p}$, the potential can be
written
\begin{align}
	V(\phi) = \frac{3 M^2 M_P^2}{4} \left(1-e^y\right)^2
\end{align}
We also have
\begin{align}
	y' = -\sqrt{\frac{2}{3}}\frac{1}{M_p}, \ \ \ y'' = 0
\end{align}
So that the derivative of the potential $\frac{dV}{d\phi}$ becomes
\begin{align}
	V' &= -\frac{3 M^2 M_p^2}{2}\left(1 - e^y\right) \cdot e^y \cdot y' \label{eq:V_diff1} \\
		 &= \sqrt{\frac{2}{3}} \frac{2}{M_p} \frac{e^y}{1-e^y} V
\end{align}
And, differentiating equation (\ref{eq:V_diff1}) once again, we find
\begin{align}
	V'' &= y' \frac{3M^2 M_p^2}{2} \frac{d}{d\phi}\left( e^{2y} - e^y \right) \\
			&= y'^2 \frac{3M^2 M_p^2}{2}\left( 2e^{2y}- e^y \right) \\
			&= \frac{2}{3}\frac{3M^2}{2}\left( 2e^{2y} - e^y \right) \\
			&= \frac{4}{3}\frac{1}{M_p^2}\frac{\left(2e^{2y} - e^y\right)}{\left( 1 - e^y\right)^2} V
\end{align}
We can use these to find the slow roll parameters. Using $M_p = \frac{E_p^2}{8\pi}$,
the slow-roll parameters are
\begin{align}
	\epsilon &= \frac{M_p^2}{2} \left(\frac{V'}{V}\right)^2 \\
  \eta 		 &= M_p^2 \frac{V''}{V}
\end{align}
By insertion, we then get
\begin{align}
	\epsilon &= \frac{4}{3}\frac{e^{2y}}{\left( 1 - e^y \right)^2} \\
  \eta 		 &= \frac{4}{3} \frac{\left(2e^{2y} - e^y\right)}{\left( 1 - e^y\right)^2}
\end{align}
\section{Problem m)}
The governing equations are then solved just like before, only switching the
quadratic potential with the Starobinsky potential. The resulting dimensionless
field $\psi$ is then shown in Figure (\ref{fig:Starobinsky_psi}), and the scale
factor can be seen in Figure (\ref{fig:Starobinsky_a}).
\\ \\
From Figure (\ref{fig:Starobinsky_a}) we note that the logarithm of the scale
factor goes as
\begin{align}
	\ln\left( \frac{a}{a_i}\right) = \tau
\end{align}
Giving an exponential scale factor during inflation
\begin{align}
	a = a_i e^\tau = a_i e^{H_i t}
\end{align}
In other words, the Starobinsky potential gives rise to inflation following
the de Sitter model.
\begin{figure}[h!]
	\includegraphics[width=\linewidth]{StarobinskyPotential_field-value.png}
	\caption{The value of the dimensionless field $\psi$ for Starobinsky inflation,
	plotted as a function of the dimensionless time $\tau$.}
	\label{fig:Starobinsky_psi}
\end{figure}

\begin{figure}[h!]
	\includegraphics[width=\linewidth]{StarobinskyPotential_scale-factor.png}
	\caption{The logarithm of the scale factor $\ln\left(\frac{a}{a_i}\right)$
	for Starobinsky inflation. Plotted against the dimensionless time $\tau$.}
	\label{fig:Starobinsky_a}
\end{figure}

\section{Problem n)}
The slow-roll parameters plotted against $N$ are shown in Figure
(\ref{fig:starobinsky_slowroll-N}), along with the slow-roll approximations
calculated in problem o). The predicted curve in the $n$-$r$ plane for
$N\in[50,60]$ can be seen in Figure (\ref{fig:starobinsky_slowroll-nr}), also
with the slow-roll approximations.

\begin{figure}[h!]
	\includegraphics[width=\linewidth]{StarobinskyPotential_slowroll-N.png}
	\caption{The slow-roll parameters $\epsilon$ and $\eta$ plotted against
	$N$, the number $e$-folds remaining before the end of inflation.}
	\label{fig:starobinsky_slowroll-N}
\end{figure}


\begin{figure}[h!]
	\includegraphics[width=\linewidth]{StarobinskyPotential_slowroll-nr.png}
	\caption{The spectral index $n$ and the tensor-to-scalar ratio of the
	Starobinsky potential, for $N\in[50,60]$.}
	\label{fig:starobinsky_slowroll-nr}
\end{figure}

\section{Problem o)}
\subsubsection{Remaining e-folds $N$}
From earlier, we have that
\begin{align}
	\frac{V}{V'} &= \sqrt{\frac{3}{2}}\frac{M_p}{2} \frac{1-e^y}{e^y} \\
							 &= \sqrt{\frac{3}{16\pi}}\frac{E_p}{2} \bigg( e^{-y} -1 \bigg) \\
\end{align}
Using $y=-\sqrt{\frac{16\pi}{3}} \frac{\phi}{E_p}$, the remaining number of
e-foldings becomes
\begin{align}
	N &= \frac{8\pi}{E_p^2} \int_{\phi_{\mathrm{end}}}^\phi \frac{V}{V'}\ d\phi \\
		&= -\frac{8\pi}{E_p} \sqrt{\frac{3}{16\pi}}\int_{y_{\mathrm{end}}}^y \frac{V}{V'}\ d\tilde{y} \\
		&= -\frac{3}{4}\int_{y_{\mathrm{end}}}^y e^{-\tilde{y}} -1\ d\tilde{y} \\
		&= \frac{3}{4}\bigg[ e^{-\tilde{y}} + \tilde{y}\bigg]_{\tilde{y}=y_{\mathrm{end}}}^y
\end{align}
Where we on the last line applied the slow-roll criterion
\begin{align}
	\phi \gg \phi_{\mathrm{end}} \ \ \ \Rightarrow \ \ \ y \gg y_{\mathrm{end}}
\end{align}
Such that
\begin{align}
 N &\approx \frac{3}{4}\bigg( e^{-y} + y\bigg)
\end{align}
Assuming the field is on the order $\phi\sim\phi_i = 2$, we have $y \approx -8$,
giving
\begin{align}
	e^{-y} \gg y
\end{align}
Which we use to further approximate $N$ as
\begin{align}
	N \approx \frac{3}{4} e^{-y}
\end{align}

\subsubsection{Slow-roll parameters $\epsilon$ \& $\eta$}
The slow-roll parameters are found similarly. Using
\begin{align}
	e^y \ll 1	\label{eq:ey_gg}
\end{align}
we have that $1-e^y\approx 1$, so that $\epsilon$ can be approximated as
\begin{align}
	\epsilon = \frac{4}{3} \frac{e^{2y}}{(1-e^y)^2}
		 \approx \frac{4}{3} e^{2y}
		 = \frac{3}{4}\frac{1}{N^2}
\end{align}
Similarly, $\eta$ becomes
\begin{align}
	\eta = \frac{4}{3} \frac{(2e^{2y} - e^y)}{(1-e^y)^2}
		 \approx \frac{4}{3} (2e^{2y} - e^y)
\end{align}
Multiplying equation (\ref{eq:ey_gg}) by $e^y$ on both sides, we get
\begin{align}
	e^{2y} \ll e^y
\end{align}
and $\eta$ is further simplified to
\begin{align}
	\eta \approx -\frac{4}{3}e^y = -\frac{1}{N}
\end{align}

\subsubsection{Tensor-to-scalar ratio $r$ \& spectral index $n$}
To approximate the tensor-to-scalar ratio $r$ we insert our approximation
of $\epsilon$, giving
\begin{align}
	r = 16\epsilon \approx \frac{12}{N^2}
\end{align}
Similarly, the spectral index is approximated
\begin{align}
	n = 1 - 6\epsilon + 2\eta &\approx 1 - \frac{9}{2}\frac{1}{N^2} - \frac{2}{N} \\
														&\approx 1 - \frac{2}{N}
\end{align}
where we used that $N$ will be a large number during slow-roll, making the
second-order term negligible.
\\
\section{Problem p)}
Analysis of the 2018 Planck CMB anisotropy measurements (Ref. \cite{Planck})
gives constraints on the values of the spectral index and the
tensor-to-scalar ratio. Specifically, they suggest a spectral index
$n = 0.9649 \pm 0.0042$ and put an upper limit $r<0.056<<$ on the tensor-to-scalar
ration.
\\ \\
While the predicted spectral index of the $\phi^2$-potential (Figure
\ref{fig:quadratic_slowroll-nr}) is consistent with this, the
predicted values of the tensor-to-scalar ratio are far above the upper limit,
ruling this potential out.
\\ \\
On the other hand, the predictions made by the Starobinsky model (Figure
\ref{fig:starobinsky_slowroll-nr}) are fully consistent with these
constraints. We see that this model predicts a spectral index within the
range observed range, and tensor-to-scalar ratios well below
the upper limit.

\section*{References}

\begin{itemize}
	\bibitem{Planck}
	Planck Collaboration, 2020,
	\textit{Planck 2018 results - X. Constraints on inflation}.
	\textit{A\&A} \textbf{641} A10 \\
	\url{https://doi.org/10.1051/0004-6361/201833887}
\end{itemize}

\end{document}
