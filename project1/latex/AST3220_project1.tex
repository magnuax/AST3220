\documentclass[reprint,english,notitlepage]{revtex4-1}  % defines the basic parameters of the document

% if you want a single-column, remove reprint

% allows special characters (including æøå)
\usepackage[utf8]{inputenc}
\usepackage[english]{babel}

%% note that you may need to download some of these packages manually, it depends on your setup.
%% I recommend downloading TeXMaker, because it includes a large library of the most common packages.

\usepackage{physics,amssymb}  % mathematical symbols (physics imports amsmath)
\usepackage{graphicx}         % include graphics such as plots
\graphicspath{{figs/}} %Setting the graphicspath

\usepackage{xcolor}           % set colors
\usepackage{hyperref}         % automagic cross-referencing (this is GODLIKE)
\usepackage{tikz}             % draw figures manually
\usepackage{listings}         % display code
\usepackage{subfigure}        % imports a lot of cool and useful figure commands

% defines the color of hyperref objects
% Blending two colors:  blue!80!black  =  80% blue and 20% black
\hypersetup{ % this is just my personal choice, feel free to change things
    colorlinks,
    linkcolor={red!50!black},
    citecolor={blue!50!black},
    urlcolor={blue!80!black}}

%% Defines the style of the programming listing
%% This is actually my personal template, go ahead and change stuff if you want
\lstset{ %
	inputpath=,
	backgroundcolor=\color{white!88!black},
	basicstyle={\ttfamily\scriptsize},
	commentstyle=\color{magenta},
	language=Python,
	morekeywords={True,False},
	tabsize=4,
	stringstyle=\color{green!55!black},
	frame=single,
	keywordstyle=\color{blue},
	showstringspaces=false,
	columns=fullflexible,
	keepspaces=true}


%% USEFUL LINKS:
%%
%%   UiO LaTeX guides:        https://www.mn.uio.no/ifi/tjenester/it/hjelp/latex/
%%   mathematics:             https://en.wikibooks.org/wiki/LaTeX/Mathematics

%%   PHYSICS !                https://mirror.hmc.edu/ctan/macros/latex/contrib/physics/physics.pdf

%%   the basics of Tikz:       https://en.wikibooks.org/wiki/LaTeX/PGF/TikZ
%%   all the colors!:          https://en.wikibooks.org/wiki/LaTeX/Colors
%%   how to draw tables:       https://en.wikibooks.org/wiki/LaTeX/Tables
%%   code listing styles:      https://en.wikibooks.org/wiki/LaTeX/Source_Code_Listings
%%   \includegraphics          https://en.wikibooks.org/wiki/LaTeX/Importing_Graphics
%%   learn more about figures  https://en.wikibooks.org/wiki/LaTeX/Floats,_Figures_and_Captions
%%   automagic bibliography:   https://en.wikibooks.org/wiki/LaTeX/Bibliography_Management  (this one is kinda difficult the first time)
%%   REVTeX Guide:             http://www.physics.csbsju.edu/370/papers/Journal_Style_Manuals/auguide4-1.pdf
%%
%%   (this document is of class "revtex4-1", the REVTeX Guide explains how the class works)


%% CREATING THE .pdf FILE USING LINUX IN THE TERMINAL
%%
%% [terminal]$ pdflatex template.tex
%%
%% Run the command twice, always.
%% If you want to use \footnote, you need to run these commands (IN THIS SPECIFIC ORDER)
%%
%% [terminal]$ pdflatex template.tex
%% [terminal]$ bibtex template
%% [terminal]$ pdflatex template.tex
%% [terminal]$ pdflatex template.tex
%%
%% Don't ask me why, I don't know.

\begin{document}
\title{AST3220 - Project 1}   % self-explanatory
\date{\today}                             % self-explanatory
\noaffiliation                            % ignore this
\begin{abstract}                          % marks the beginning of the abstract
This abstract is abstract.                % the body of the abstract
\end{abstract}                            % marks the end of the abstract
\maketitle                                % creates the title, author, date & abstract

% the fundamental components of scientific reports:
The scalar field has energy density and pressure
\begin{align}
	\rho_{\phi} = \frac{1}{2} \dot{\phi}^2 + V(\phi) \label{eq:density} \\
	   p_{\phi} = \frac{1}{2} \dot{\phi}^2 - V(\phi) \label{eq:pressure}
\end{align}
\section{Problem 1}
Assuming the quintessence field follows the continuity equation
\begin{align}
	\dot{\rho}_{\phi} &= -3H\left(\rho_{\phi} + p_{\phi}\right) \\
										&= -3\frac{\dot{a}}{a}\left(1 + w_{\phi}\right)\rho_{\phi}
\end{align}
we can begin solving for the density by separating the variables:
\begin{align}\label{eq:drho}
	\frac{1}{\rho_{\phi}} d\rho_{\phi} = -\frac{3}{a}\left(1+w_{\phi}\right)da
\end{align}
We then rewrite $a(t)$ in terms of the cosmological redshift of light emitted at some point $t$ in the past:
\begin{align}
	1+z = \frac{a_0}{a(t)} \ &\Rightarrow a(t) = \frac{a_0}{1+z} \\
													 &\Rightarrow da  = -\frac{a_0}{(1+z)^2}dz
\end{align}
Inserting these into eq. \ref{eq:drho} and integrating from some time $t$ to
today, we get:
\begin{align}
	\int^{\rho_{\phi0}}_{\rho_{\phi}} d\rho_{\phi} \frac{1}{\rho_{\phi}}
	= \int^{0}_z dz' \frac{3[1+w_\phi (z')]}{(1+z')}
\end{align}
We then flip the integration limits on both sides, canceling the negatives.
Computing the left hand integral and solving for $rho_\phi$ we then get the solution
\begin{align}
	\rho_\phi(z) = \rho_{\phi 0} \
	\mathrm{exp}\left\{ \int^{z}_0 dz' \frac{3[1+w_\phi (z')]}{(1+z')}\right\}
\end{align}

\section{Problem 2}
Inserting equations for the density (\ref{eq:density}) and pressure
(\ref{eq:pressure}) of the scalar field into the continuity equation we get
\begin{align}
	\dot{\rho}_\phi &= -3H\left(\rho_\phi + p_\phi \left) \\
									&= -3H\dot{\phi}^2
\end{align}
We can also find an expression for $\dot{\rho}_\phi$ directly, by taking the
time derivative of equation (\ref{eq:density}):
\begin{align}
	\dot{\rho}_\phi &= \frac{d}{dt}\left(\frac{1}{2} \dot{\phi}^2 + V(\phi) \right)\\
					  			&= \dot{\phi} \ddot{\phi} + {V}'(\phi) \dot{\phi}
\end{align}
Equating the two expressions, we get the differential equation:
\begin{align}
	\ddot{\phi}\dot{\phi}  + 3H\dot{\phi}^2 + {V}'(\phi) \dot{\phi} = 0
\end{align}
Ignoring the boring case of a static $\phi(t)=\phi_0$, we have $\dot{\phi} \neq0$.
The equation can then be reduced to:
\begin{align}
	\ddot{\phi} + 3H\dot{\phi} + {V}'(\phi) = 0 \label{eq:phi_ddot}
\end{align}

\section{Problem 3}
Taking the time-derivative of the Hubble parameter
\begin{align}
	\dot{H} = \frac{d}{dt}\left(\frac{\dot{a}}{a}\right) = \frac{\ddot{a}}{a} - \left(\frac{\dot{a}}{a}\right)^2 \label{eq:Hdot}
\end{align}
we recognize both terms on the right hand side from the Friedmann equations. For a flat ($k=0$) universe the Friedmann equations read
\begin{align}
	\left(\frac{\dot{a}}{a}\right)^2 &= \frac{\kappa^2}{3}\rho \\
	\frac{\ddot{a}}{a}		&= -\frac{\kappa^2}{6}\left(\rho + 3p \right)
\end{align}
Where we have defined $\kappa^2 = 8\pi G$. For a universe with matter, radiation
and quintessence we have
\begin{align}
	\rho &= \rho_m + \rho_r + \rho_\phi \\
	p    &= p_r + p_\phi
\end{align}
Inserting the Friedmann equations into (\ref{eq:Hdot}), and writing the pressures in terms of the equation of state parameter, we get
\begin{align}
	\dot{H} &= -\frac{\kappa^2}{6}\left(\rho + 3p \right) - \frac{\kappa^2}{3}\rho \\
					&= -\frac{\kappa^2}{2} \left[\rho + p \right] \\
					&= -\frac{\kappa^2}{2} \left[ \rho_m + \rho_r(1+w_r) + \dot{\phi}\right]\\
					&= -\frac{\kappa^2}{2} \left[ \rho_m + \rho_r(1+w_r) + \dot{\phi}\right]
					\label{eq:Hdot}
\end{align}

\section{Problem 4}
We introduce the dimensionless variables
\begin{align}
	x_1 = \frac{\kappa \dot{\phi}}{\sqrt{6}H} \\
	x_2 = \frac{\kappa \sqrt{V}}{\sqrt{3}H} \\
	x_3 = \frac{\kappa \sqrt{\rho_r}}{\sqrt{3}H} \\
\end{align}
and notice their squares can be rewritten them in terms of the critical density
$\rho_c = \frac{3H^2}{\kappa^2}$:
\begin{align}
	x_1^2 &= \frac{\frac{1}{2}\dot{\phi}^2}{\rho_c} \label{eq:x1} \\
	x_2^2 &= \frac{V}{\rho_c} \label{eq:x2} \\
	x_3^2 &= \frac{\rho_r}{\rho_c} \label{eq:x3} \\
\end{align}
The quintessence and radiation density parameters can then easily be found to be
\begin{align}
	\Omega_\phi &= \frac{\frac{1}{2}\dot{\phi} + V}{\rho_c} = x_1^2 + x_2^2 \\
	\Omega_r &= \frac{\rho_r}{\rho_c} = x_3^2
\end{align}
Lastly we want to find an expression for the density parameter for
mass. We accomplish this by taking advantage of our model being
spatially flat, meaning $\frac{\rho}{\rho_c} = 1$. Expanding $\rho$
we get
\begin{align}
	1 &= \frac{\rho_\phi + \rho_r + \rho_m}{\rho_c} \\
		&= \Omega_\phi + \Omega_r + \Omega_m
\end{align}
Which we can solve for $\Omega_m$:
\begin{align}
	\Omega_m &= 1 - \Omega_\phi - \Omega_m \\
					 &= 1 - x_1^2 - x_2^2 - x_3^2
\end{align}

\section{Problem 5}
Next we want to rewrite equation (\ref{eq:Hdot}) in terms of the dimensionless
variables. Using the definitions of the density parameters and the dimensionless
variables we get:
\begin{align}
	\dot{H} &= -\frac{\kappa^2}{2} \left[ \rho_m + \rho_r(1+w_r) + \dot{\phi}\right] \\
					&= -\frac{\kappa^2 \rho_c}{2}\left[\Omega_m + \Omega_r(1+w_r) + 2\Omega_\phi -2x_2^2\right] \\
					&= -\frac{3H^2}{2}\left[1 + x_1^2 -x_2^2 -x_3^2 + x_3^2(1+w_r)\right]
\end{align}
Simplyfing and inserting the radiation equation of state parameter $w_r = 1/3$,
we get:
\begin{align}
	\frac{\dot{H}}{H^2} &= -\frac{1}{2}\left[3+3x_1^2 -3x_2^2 + x_3^2\right] \label{eq:H_dim}
\end{align}

\section{Problem 6}
\subsection{Equation of motion for $x_1$}
Using the product rule:
\begin{align}
	\frac{dx_1}{dN} &= \frac{1}{H}\frac{dx_1}{dt} \\
	&= \frac{\kappa}{\sqrt{6}H}\frac{d}{dt}\left(\frac{\dot{\phi}}{H}\right)\\
	&= \frac{\kappa\ddot{\phi}}{\sqrt{6}H^2} - \frac{\kappa\dot{\phi}}{\sqrt{6}H}\frac{\dot{H}}{H^2}
\end{align}
From equation (\ref{eq:phi_ddot}) we have $\ddot{\phi}=-3H\dot{\phi}-{V}$.
Inserting for $\ddot{\phi}$ and $\dot{H}/H^2$ (eq. \ref{eq:H_dim}) we then get:
\begin{align}
	\frac{dx_1}{dN} &= -\frac{3\kappa\dot{\phi}}{\sqrt{6}H} - \frac{\kappa V'}{\sqrt{6}H}
										+ \frac{\kappa\dot{\phi}}{\sqrt{6}H}\frac{\dot{H}}{H} \\
									&= -3x_1	 + \frac{\sqrt{6}}{2}\lambda x_2^2
										+ \frac{1}{2}x_1(3+3x_1 - 3x_2^2 + 3x_3^3)
\end{align}
Where we have defined $\lambda$ as
\begin{align}
	\lambda = -\frac{V'}{\kappa V}
\end{align}

\subsection{Equation of motion for $x_2$}
The equation for $x_2$ is found in a similar manner, we get:
\begin{align}
	\frac{dx_2}{dN} &= \frac{\kappa}{\sqrt{3}H}\frac{d}{dt}\left(\frac{\sqrt{V}}{H}\right)\\
									&= \frac{\kappa}{\sqrt{3}H}\left(\frac{1}{2}\frac{V'\dot{\phi}}
									{\sqrt{\phi}H}-\sqrt{V}\frac{\dot{H}}{H^2}\right) \\
									&=\frac{\sqrt{6}}{2}\frac{V'}{\kappa V}\frac{\kappa \sqrt{V}}{\sqrt{3}H}
										\frac{\kappa\dot{\phi}}{\sqrt{6}H} - \frac{\kappa\sqrt{V}}{\sqrt{3}H}
										\frac{\dot{H}}{H^2} \\
									&=\frac{\sqrt{6}}{2}\lambda x_1 x_2 + \frac{1}{2}x_2
									(3+3x_1 - 3x_2^2 + 3x_3^3)
\end{align}
\subsection{Equation of motion for $x_3$}
Similarly, for $x_3$:

\begin{align}
	\frac{dx_3}{dN} &= \frac{\kappa}{\sqrt{3}H}\frac{d}{dt}\left(\frac{\sqrt{\rho_r}}{H}\right)\\
									&= \frac{\kappa}{\sqrt{3}H}\left(\frac{1}{2}\frac{\dot{\rho}_r}
									{\sqrt{\rho_r}H} - \sqrt{\rho_r}\frac{\dot{H}}{H^2}\right)\\
\end{align}
Where $\dot{\rho}_r$ is given by the first Friedmann equation:
\begin{align}
	\dot{\rho}_r &= -3 H (1 + w_r)\rho_r \\
							 &= -4 H\rho_r
\end{align}
Where we used that $w_r=1/3$. Inserting the expressions for $\dot{\rho}_r$ and
$\dot{H}/H^2$ we finally get:
\begin{align}
	\frac{dx_3}{dN} &=-2x_3 + \frac{1}{2}x_3\left(3+3x_1 - 3x_2^2 + 3x_3^3\right)
\end{align}

\section{Problem 7}
Rewriting the expression for $\lambda$ as a differential equation for V, we get:
\begin{align}
	V' + \kappa \lambda V = 0
\end{align}
For a constant $\lambda$ this has the solution
\begin{align}
	V = V_0 e^{-\kappa\lambda\phi}
\end{align}
We can then find the value of $\Gamma$ by simple differentiation:
\begin{align}
	\Gamma &= \frac{V V''}{(V')^2} \\
				 &= \frac{\kappa^2\lambda^2 V^2}{(-\kappa\lambda)^2 V^2} \\
				 &= 1
\end{align}
\section{Problem 8}
For a non-constant $\lambda$
\section{Problem 9}
CALCULATE SUM OF DENSITY PARAMETERS FOR QUALITY CONTROL!

\begin{acknowledgments}  % if you disagree with the spelling, blame Americans
I would like thank myself for writing this beautiful document.
\end{acknowledgments}


%% When it comes to the bibliography I personally generate it using BibLaTeX. (see the link above if you're interested)
%% You're obviously allowed to create the references section however you like.
%% I'll keep it simple here.
\section*{References}  % the asterisk (*) after \section makes the section numbering go away
\begin{itemize}
\item[-]Reference 1
\item[-]Reference 2
\end{itemize}

\newpage
%% if you want to include an appendix, this is how you do it
\appendix
\section{Name of appendix}
This will be the body of the appendix.
\section{This is another appendix}\label{appendix}
Tada.
%% all \section commands following \appendix are automatically taken as appendices

%% Note that \label{appendix} command on line 115. What this does is setup a reference point for LaTeX that you can
%% access wherever you want using \ref{appendix}.
%% You can place labels on most environments such as equations, figures, tables, etc.


%% If you want to include figure:
%\includegraphics[scale=1.0]{filename}
%% check https://en.wikibooks.org/wiki/LaTeX/Importing_Graphics if you want to know more

\end{document}
